
% Default to the notebook output style

    


% Inherit from the specified cell style.




    
\documentclass[11pt]{article}

    
    
    \usepackage[T1]{fontenc}
    % Nicer default font (+ math font) than Computer Modern for most use cases
    \usepackage{mathpazo}

    % Basic figure setup, for now with no caption control since it's done
    % automatically by Pandoc (which extracts ![](path) syntax from Markdown).
    \usepackage{graphicx}
    % We will generate all images so they have a width \maxwidth. This means
    % that they will get their normal width if they fit onto the page, but
    % are scaled down if they would overflow the margins.
    \makeatletter
    \def\maxwidth{\ifdim\Gin@nat@width>\linewidth\linewidth
    \else\Gin@nat@width\fi}
    \makeatother
    \let\Oldincludegraphics\includegraphics
    % Set max figure width to be 80% of text width, for now hardcoded.
    \renewcommand{\includegraphics}[1]{\Oldincludegraphics[width=.8\maxwidth]{#1}}
    % Ensure that by default, figures have no caption (until we provide a
    % proper Figure object with a Caption API and a way to capture that
    % in the conversion process - todo).
    \usepackage{caption}
    \DeclareCaptionLabelFormat{nolabel}{}
    \captionsetup{labelformat=nolabel}

    \usepackage{adjustbox} % Used to constrain images to a maximum size 
    \usepackage{xcolor} % Allow colors to be defined
    \usepackage{enumerate} % Needed for markdown enumerations to work
    \usepackage{geometry} % Used to adjust the document margins
    \usepackage{amsmath} % Equations
    \usepackage{amssymb} % Equations
    \usepackage{textcomp} % defines textquotesingle
    % Hack from http://tex.stackexchange.com/a/47451/13684:
    \AtBeginDocument{%
        \def\PYZsq{\textquotesingle}% Upright quotes in Pygmentized code
    }
    \usepackage{upquote} % Upright quotes for verbatim code
    \usepackage{eurosym} % defines \euro
    \usepackage[mathletters]{ucs} % Extended unicode (utf-8) support
    \usepackage[utf8x]{inputenc} % Allow utf-8 characters in the tex document
    \usepackage{fancyvrb} % verbatim replacement that allows latex
    \usepackage{grffile} % extends the file name processing of package graphics 
                         % to support a larger range 
    % The hyperref package gives us a pdf with properly built
    % internal navigation ('pdf bookmarks' for the table of contents,
    % internal cross-reference links, web links for URLs, etc.)
    \usepackage{hyperref}
    \usepackage{longtable} % longtable support required by pandoc >1.10
    \usepackage{booktabs}  % table support for pandoc > 1.12.2
    \usepackage[inline]{enumitem} % IRkernel/repr support (it uses the enumerate* environment)
    \usepackage[normalem]{ulem} % ulem is needed to support strikethroughs (\sout)
                                % normalem makes italics be italics, not underlines
    \usepackage{mathrsfs}
    

    
    
    % Colors for the hyperref package
    \definecolor{urlcolor}{rgb}{0,.145,.698}
    \definecolor{linkcolor}{rgb}{.71,0.21,0.01}
    \definecolor{citecolor}{rgb}{.12,.54,.11}

    % ANSI colors
    \definecolor{ansi-black}{HTML}{3E424D}
    \definecolor{ansi-black-intense}{HTML}{282C36}
    \definecolor{ansi-red}{HTML}{E75C58}
    \definecolor{ansi-red-intense}{HTML}{B22B31}
    \definecolor{ansi-green}{HTML}{00A250}
    \definecolor{ansi-green-intense}{HTML}{007427}
    \definecolor{ansi-yellow}{HTML}{DDB62B}
    \definecolor{ansi-yellow-intense}{HTML}{B27D12}
    \definecolor{ansi-blue}{HTML}{208FFB}
    \definecolor{ansi-blue-intense}{HTML}{0065CA}
    \definecolor{ansi-magenta}{HTML}{D160C4}
    \definecolor{ansi-magenta-intense}{HTML}{A03196}
    \definecolor{ansi-cyan}{HTML}{60C6C8}
    \definecolor{ansi-cyan-intense}{HTML}{258F8F}
    \definecolor{ansi-white}{HTML}{C5C1B4}
    \definecolor{ansi-white-intense}{HTML}{A1A6B2}
    \definecolor{ansi-default-inverse-fg}{HTML}{FFFFFF}
    \definecolor{ansi-default-inverse-bg}{HTML}{000000}

    % commands and environments needed by pandoc snippets
    % extracted from the output of `pandoc -s`
    \providecommand{\tightlist}{%
      \setlength{\itemsep}{0pt}\setlength{\parskip}{0pt}}
    \DefineVerbatimEnvironment{Highlighting}{Verbatim}{commandchars=\\\{\}}
    % Add ',fontsize=\small' for more characters per line
    \newenvironment{Shaded}{}{}
    \newcommand{\KeywordTok}[1]{\textcolor[rgb]{0.00,0.44,0.13}{\textbf{{#1}}}}
    \newcommand{\DataTypeTok}[1]{\textcolor[rgb]{0.56,0.13,0.00}{{#1}}}
    \newcommand{\DecValTok}[1]{\textcolor[rgb]{0.25,0.63,0.44}{{#1}}}
    \newcommand{\BaseNTok}[1]{\textcolor[rgb]{0.25,0.63,0.44}{{#1}}}
    \newcommand{\FloatTok}[1]{\textcolor[rgb]{0.25,0.63,0.44}{{#1}}}
    \newcommand{\CharTok}[1]{\textcolor[rgb]{0.25,0.44,0.63}{{#1}}}
    \newcommand{\StringTok}[1]{\textcolor[rgb]{0.25,0.44,0.63}{{#1}}}
    \newcommand{\CommentTok}[1]{\textcolor[rgb]{0.38,0.63,0.69}{\textit{{#1}}}}
    \newcommand{\OtherTok}[1]{\textcolor[rgb]{0.00,0.44,0.13}{{#1}}}
    \newcommand{\AlertTok}[1]{\textcolor[rgb]{1.00,0.00,0.00}{\textbf{{#1}}}}
    \newcommand{\FunctionTok}[1]{\textcolor[rgb]{0.02,0.16,0.49}{{#1}}}
    \newcommand{\RegionMarkerTok}[1]{{#1}}
    \newcommand{\ErrorTok}[1]{\textcolor[rgb]{1.00,0.00,0.00}{\textbf{{#1}}}}
    \newcommand{\NormalTok}[1]{{#1}}
    
    % Additional commands for more recent versions of Pandoc
    \newcommand{\ConstantTok}[1]{\textcolor[rgb]{0.53,0.00,0.00}{{#1}}}
    \newcommand{\SpecialCharTok}[1]{\textcolor[rgb]{0.25,0.44,0.63}{{#1}}}
    \newcommand{\VerbatimStringTok}[1]{\textcolor[rgb]{0.25,0.44,0.63}{{#1}}}
    \newcommand{\SpecialStringTok}[1]{\textcolor[rgb]{0.73,0.40,0.53}{{#1}}}
    \newcommand{\ImportTok}[1]{{#1}}
    \newcommand{\DocumentationTok}[1]{\textcolor[rgb]{0.73,0.13,0.13}{\textit{{#1}}}}
    \newcommand{\AnnotationTok}[1]{\textcolor[rgb]{0.38,0.63,0.69}{\textbf{\textit{{#1}}}}}
    \newcommand{\CommentVarTok}[1]{\textcolor[rgb]{0.38,0.63,0.69}{\textbf{\textit{{#1}}}}}
    \newcommand{\VariableTok}[1]{\textcolor[rgb]{0.10,0.09,0.49}{{#1}}}
    \newcommand{\ControlFlowTok}[1]{\textcolor[rgb]{0.00,0.44,0.13}{\textbf{{#1}}}}
    \newcommand{\OperatorTok}[1]{\textcolor[rgb]{0.40,0.40,0.40}{{#1}}}
    \newcommand{\BuiltInTok}[1]{{#1}}
    \newcommand{\ExtensionTok}[1]{{#1}}
    \newcommand{\PreprocessorTok}[1]{\textcolor[rgb]{0.74,0.48,0.00}{{#1}}}
    \newcommand{\AttributeTok}[1]{\textcolor[rgb]{0.49,0.56,0.16}{{#1}}}
    \newcommand{\InformationTok}[1]{\textcolor[rgb]{0.38,0.63,0.69}{\textbf{\textit{{#1}}}}}
    \newcommand{\WarningTok}[1]{\textcolor[rgb]{0.38,0.63,0.69}{\textbf{\textit{{#1}}}}}
    
    
    % Define a nice break command that doesn't care if a line doesn't already
    % exist.
    \def\br{\hspace*{\fill} \\* }
    % Math Jax compatibility definitions
    \def\gt{>}
    \def\lt{<}
    \let\Oldtex\TeX
    \let\Oldlatex\LaTeX
    \renewcommand{\TeX}{\textrm{\Oldtex}}
    \renewcommand{\LaTeX}{\textrm{\Oldlatex}}
    % Document parameters
    % Document title
    \title{1S 2018 TP suites sol}
    
    
    
    
    

    % Pygments definitions
    
\makeatletter
\def\PY@reset{\let\PY@it=\relax \let\PY@bf=\relax%
    \let\PY@ul=\relax \let\PY@tc=\relax%
    \let\PY@bc=\relax \let\PY@ff=\relax}
\def\PY@tok#1{\csname PY@tok@#1\endcsname}
\def\PY@toks#1+{\ifx\relax#1\empty\else%
    \PY@tok{#1}\expandafter\PY@toks\fi}
\def\PY@do#1{\PY@bc{\PY@tc{\PY@ul{%
    \PY@it{\PY@bf{\PY@ff{#1}}}}}}}
\def\PY#1#2{\PY@reset\PY@toks#1+\relax+\PY@do{#2}}

\expandafter\def\csname PY@tok@w\endcsname{\def\PY@tc##1{\textcolor[rgb]{0.73,0.73,0.73}{##1}}}
\expandafter\def\csname PY@tok@c\endcsname{\let\PY@it=\textit\def\PY@tc##1{\textcolor[rgb]{0.25,0.50,0.50}{##1}}}
\expandafter\def\csname PY@tok@cp\endcsname{\def\PY@tc##1{\textcolor[rgb]{0.74,0.48,0.00}{##1}}}
\expandafter\def\csname PY@tok@k\endcsname{\let\PY@bf=\textbf\def\PY@tc##1{\textcolor[rgb]{0.00,0.50,0.00}{##1}}}
\expandafter\def\csname PY@tok@kp\endcsname{\def\PY@tc##1{\textcolor[rgb]{0.00,0.50,0.00}{##1}}}
\expandafter\def\csname PY@tok@kt\endcsname{\def\PY@tc##1{\textcolor[rgb]{0.69,0.00,0.25}{##1}}}
\expandafter\def\csname PY@tok@o\endcsname{\def\PY@tc##1{\textcolor[rgb]{0.40,0.40,0.40}{##1}}}
\expandafter\def\csname PY@tok@ow\endcsname{\let\PY@bf=\textbf\def\PY@tc##1{\textcolor[rgb]{0.67,0.13,1.00}{##1}}}
\expandafter\def\csname PY@tok@nb\endcsname{\def\PY@tc##1{\textcolor[rgb]{0.00,0.50,0.00}{##1}}}
\expandafter\def\csname PY@tok@nf\endcsname{\def\PY@tc##1{\textcolor[rgb]{0.00,0.00,1.00}{##1}}}
\expandafter\def\csname PY@tok@nc\endcsname{\let\PY@bf=\textbf\def\PY@tc##1{\textcolor[rgb]{0.00,0.00,1.00}{##1}}}
\expandafter\def\csname PY@tok@nn\endcsname{\let\PY@bf=\textbf\def\PY@tc##1{\textcolor[rgb]{0.00,0.00,1.00}{##1}}}
\expandafter\def\csname PY@tok@ne\endcsname{\let\PY@bf=\textbf\def\PY@tc##1{\textcolor[rgb]{0.82,0.25,0.23}{##1}}}
\expandafter\def\csname PY@tok@nv\endcsname{\def\PY@tc##1{\textcolor[rgb]{0.10,0.09,0.49}{##1}}}
\expandafter\def\csname PY@tok@no\endcsname{\def\PY@tc##1{\textcolor[rgb]{0.53,0.00,0.00}{##1}}}
\expandafter\def\csname PY@tok@nl\endcsname{\def\PY@tc##1{\textcolor[rgb]{0.63,0.63,0.00}{##1}}}
\expandafter\def\csname PY@tok@ni\endcsname{\let\PY@bf=\textbf\def\PY@tc##1{\textcolor[rgb]{0.60,0.60,0.60}{##1}}}
\expandafter\def\csname PY@tok@na\endcsname{\def\PY@tc##1{\textcolor[rgb]{0.49,0.56,0.16}{##1}}}
\expandafter\def\csname PY@tok@nt\endcsname{\let\PY@bf=\textbf\def\PY@tc##1{\textcolor[rgb]{0.00,0.50,0.00}{##1}}}
\expandafter\def\csname PY@tok@nd\endcsname{\def\PY@tc##1{\textcolor[rgb]{0.67,0.13,1.00}{##1}}}
\expandafter\def\csname PY@tok@s\endcsname{\def\PY@tc##1{\textcolor[rgb]{0.73,0.13,0.13}{##1}}}
\expandafter\def\csname PY@tok@sd\endcsname{\let\PY@it=\textit\def\PY@tc##1{\textcolor[rgb]{0.73,0.13,0.13}{##1}}}
\expandafter\def\csname PY@tok@si\endcsname{\let\PY@bf=\textbf\def\PY@tc##1{\textcolor[rgb]{0.73,0.40,0.53}{##1}}}
\expandafter\def\csname PY@tok@se\endcsname{\let\PY@bf=\textbf\def\PY@tc##1{\textcolor[rgb]{0.73,0.40,0.13}{##1}}}
\expandafter\def\csname PY@tok@sr\endcsname{\def\PY@tc##1{\textcolor[rgb]{0.73,0.40,0.53}{##1}}}
\expandafter\def\csname PY@tok@ss\endcsname{\def\PY@tc##1{\textcolor[rgb]{0.10,0.09,0.49}{##1}}}
\expandafter\def\csname PY@tok@sx\endcsname{\def\PY@tc##1{\textcolor[rgb]{0.00,0.50,0.00}{##1}}}
\expandafter\def\csname PY@tok@m\endcsname{\def\PY@tc##1{\textcolor[rgb]{0.40,0.40,0.40}{##1}}}
\expandafter\def\csname PY@tok@gh\endcsname{\let\PY@bf=\textbf\def\PY@tc##1{\textcolor[rgb]{0.00,0.00,0.50}{##1}}}
\expandafter\def\csname PY@tok@gu\endcsname{\let\PY@bf=\textbf\def\PY@tc##1{\textcolor[rgb]{0.50,0.00,0.50}{##1}}}
\expandafter\def\csname PY@tok@gd\endcsname{\def\PY@tc##1{\textcolor[rgb]{0.63,0.00,0.00}{##1}}}
\expandafter\def\csname PY@tok@gi\endcsname{\def\PY@tc##1{\textcolor[rgb]{0.00,0.63,0.00}{##1}}}
\expandafter\def\csname PY@tok@gr\endcsname{\def\PY@tc##1{\textcolor[rgb]{1.00,0.00,0.00}{##1}}}
\expandafter\def\csname PY@tok@ge\endcsname{\let\PY@it=\textit}
\expandafter\def\csname PY@tok@gs\endcsname{\let\PY@bf=\textbf}
\expandafter\def\csname PY@tok@gp\endcsname{\let\PY@bf=\textbf\def\PY@tc##1{\textcolor[rgb]{0.00,0.00,0.50}{##1}}}
\expandafter\def\csname PY@tok@go\endcsname{\def\PY@tc##1{\textcolor[rgb]{0.53,0.53,0.53}{##1}}}
\expandafter\def\csname PY@tok@gt\endcsname{\def\PY@tc##1{\textcolor[rgb]{0.00,0.27,0.87}{##1}}}
\expandafter\def\csname PY@tok@err\endcsname{\def\PY@bc##1{\setlength{\fboxsep}{0pt}\fcolorbox[rgb]{1.00,0.00,0.00}{1,1,1}{\strut ##1}}}
\expandafter\def\csname PY@tok@kc\endcsname{\let\PY@bf=\textbf\def\PY@tc##1{\textcolor[rgb]{0.00,0.50,0.00}{##1}}}
\expandafter\def\csname PY@tok@kd\endcsname{\let\PY@bf=\textbf\def\PY@tc##1{\textcolor[rgb]{0.00,0.50,0.00}{##1}}}
\expandafter\def\csname PY@tok@kn\endcsname{\let\PY@bf=\textbf\def\PY@tc##1{\textcolor[rgb]{0.00,0.50,0.00}{##1}}}
\expandafter\def\csname PY@tok@kr\endcsname{\let\PY@bf=\textbf\def\PY@tc##1{\textcolor[rgb]{0.00,0.50,0.00}{##1}}}
\expandafter\def\csname PY@tok@bp\endcsname{\def\PY@tc##1{\textcolor[rgb]{0.00,0.50,0.00}{##1}}}
\expandafter\def\csname PY@tok@fm\endcsname{\def\PY@tc##1{\textcolor[rgb]{0.00,0.00,1.00}{##1}}}
\expandafter\def\csname PY@tok@vc\endcsname{\def\PY@tc##1{\textcolor[rgb]{0.10,0.09,0.49}{##1}}}
\expandafter\def\csname PY@tok@vg\endcsname{\def\PY@tc##1{\textcolor[rgb]{0.10,0.09,0.49}{##1}}}
\expandafter\def\csname PY@tok@vi\endcsname{\def\PY@tc##1{\textcolor[rgb]{0.10,0.09,0.49}{##1}}}
\expandafter\def\csname PY@tok@vm\endcsname{\def\PY@tc##1{\textcolor[rgb]{0.10,0.09,0.49}{##1}}}
\expandafter\def\csname PY@tok@sa\endcsname{\def\PY@tc##1{\textcolor[rgb]{0.73,0.13,0.13}{##1}}}
\expandafter\def\csname PY@tok@sb\endcsname{\def\PY@tc##1{\textcolor[rgb]{0.73,0.13,0.13}{##1}}}
\expandafter\def\csname PY@tok@sc\endcsname{\def\PY@tc##1{\textcolor[rgb]{0.73,0.13,0.13}{##1}}}
\expandafter\def\csname PY@tok@dl\endcsname{\def\PY@tc##1{\textcolor[rgb]{0.73,0.13,0.13}{##1}}}
\expandafter\def\csname PY@tok@s2\endcsname{\def\PY@tc##1{\textcolor[rgb]{0.73,0.13,0.13}{##1}}}
\expandafter\def\csname PY@tok@sh\endcsname{\def\PY@tc##1{\textcolor[rgb]{0.73,0.13,0.13}{##1}}}
\expandafter\def\csname PY@tok@s1\endcsname{\def\PY@tc##1{\textcolor[rgb]{0.73,0.13,0.13}{##1}}}
\expandafter\def\csname PY@tok@mb\endcsname{\def\PY@tc##1{\textcolor[rgb]{0.40,0.40,0.40}{##1}}}
\expandafter\def\csname PY@tok@mf\endcsname{\def\PY@tc##1{\textcolor[rgb]{0.40,0.40,0.40}{##1}}}
\expandafter\def\csname PY@tok@mh\endcsname{\def\PY@tc##1{\textcolor[rgb]{0.40,0.40,0.40}{##1}}}
\expandafter\def\csname PY@tok@mi\endcsname{\def\PY@tc##1{\textcolor[rgb]{0.40,0.40,0.40}{##1}}}
\expandafter\def\csname PY@tok@il\endcsname{\def\PY@tc##1{\textcolor[rgb]{0.40,0.40,0.40}{##1}}}
\expandafter\def\csname PY@tok@mo\endcsname{\def\PY@tc##1{\textcolor[rgb]{0.40,0.40,0.40}{##1}}}
\expandafter\def\csname PY@tok@ch\endcsname{\let\PY@it=\textit\def\PY@tc##1{\textcolor[rgb]{0.25,0.50,0.50}{##1}}}
\expandafter\def\csname PY@tok@cm\endcsname{\let\PY@it=\textit\def\PY@tc##1{\textcolor[rgb]{0.25,0.50,0.50}{##1}}}
\expandafter\def\csname PY@tok@cpf\endcsname{\let\PY@it=\textit\def\PY@tc##1{\textcolor[rgb]{0.25,0.50,0.50}{##1}}}
\expandafter\def\csname PY@tok@c1\endcsname{\let\PY@it=\textit\def\PY@tc##1{\textcolor[rgb]{0.25,0.50,0.50}{##1}}}
\expandafter\def\csname PY@tok@cs\endcsname{\let\PY@it=\textit\def\PY@tc##1{\textcolor[rgb]{0.25,0.50,0.50}{##1}}}

\def\PYZbs{\char`\\}
\def\PYZus{\char`\_}
\def\PYZob{\char`\{}
\def\PYZcb{\char`\}}
\def\PYZca{\char`\^}
\def\PYZam{\char`\&}
\def\PYZlt{\char`\<}
\def\PYZgt{\char`\>}
\def\PYZsh{\char`\#}
\def\PYZpc{\char`\%}
\def\PYZdl{\char`\$}
\def\PYZhy{\char`\-}
\def\PYZsq{\char`\'}
\def\PYZdq{\char`\"}
\def\PYZti{\char`\~}
% for compatibility with earlier versions
\def\PYZat{@}
\def\PYZlb{[}
\def\PYZrb{]}
\makeatother


    % Exact colors from NB
    \definecolor{incolor}{rgb}{0.0, 0.0, 0.5}
    \definecolor{outcolor}{rgb}{0.545, 0.0, 0.0}



    
    % Prevent overflowing lines due to hard-to-break entities
    \sloppy 
    % Setup hyperref package
    \hypersetup{
      breaklinks=true,  % so long urls are correctly broken across lines
      colorlinks=true,
      urlcolor=urlcolor,
      linkcolor=linkcolor,
      citecolor=citecolor,
      }
    % Slightly bigger margins than the latex defaults
    
    \geometry{verbose,tmargin=1in,bmargin=1in,lmargin=1in,rmargin=1in}
    
    

    \begin{document}
    
    
    \maketitle
    
    

    
    \section{TP INFO EN ACCOMPAGNEMENT DU
TD03}\label{tp-info-en-accompagnement-du-td03}

    \subsection{1. En respectant les indications (afficher les termes
jusqu'à celui de rang
10)}\label{en-respectant-les-indications-afficher-les-termes-jusquuxe0-celui-de-rang-10}

\begin{quote}
(\textbf{1.1.2}) Suite \((w_n)\) telle que, pour tout
\(n\in\mathbb{N}\), \(w_n=\dfrac{2n-3}{n+2}\)
\end{quote}

    \begin{Verbatim}[commandchars=\\\{\}]
{\color{incolor}In [{\color{incolor}1}]:} \PY{k}{for} \PY{n}{i} \PY{o+ow}{in} \PY{n+nb}{range}\PY{p}{(}\PY{l+m+mi}{11}\PY{p}{)}\PY{p}{:}   \PY{c+c1}{\PYZsh{} commentaire : changer range(10) en range(11) }
                              \PY{c+c1}{\PYZsh{} pour afficher le terme de rang 10}
            \PY{n}{w}\PY{o}{=}\PY{p}{(}\PY{l+m+mi}{2}\PY{o}{*}\PY{n}{i}\PY{o}{\PYZhy{}}\PY{l+m+mi}{3}\PY{p}{)}\PY{o}{/}\PY{p}{(}\PY{n}{i}\PY{o}{+}\PY{l+m+mi}{2}\PY{p}{)}
            \PY{n+nb}{print}\PY{p}{(}\PY{n}{i}\PY{p}{,}\PY{n}{w}\PY{p}{)}
\end{Verbatim}

    \begin{Verbatim}[commandchars=\\\{\}]
0 -1.5
1 -0.3333333333333333
2 0.25
3 0.6
4 0.8333333333333334
5 1.0
6 1.125
7 1.2222222222222223
8 1.3
9 1.3636363636363635
10 1.4166666666666667

    \end{Verbatim}

    \begin{quote}
(\textbf{1.2.2}) Suite \((y_n)\), telle que \(y_1=-3\) et pour tout
\(n\geq 1\), \(y_{n+1}=y_n+2n\)
\end{quote}

    \begin{Verbatim}[commandchars=\\\{\}]
{\color{incolor}In [{\color{incolor}2}]:} \PY{n}{y}\PY{o}{=}\PY{o}{\PYZhy{}}\PY{l+m+mi}{3}
        \PY{k}{for} \PY{n}{i} \PY{o+ow}{in} \PY{n+nb}{range}\PY{p}{(}\PY{l+m+mi}{10}\PY{p}{)}\PY{p}{:}
            \PY{n}{y}\PY{o}{=}\PY{n}{y}\PY{o}{+}\PY{l+m+mi}{2}\PY{o}{*}\PY{n}{i}
            \PY{n+nb}{print}\PY{p}{(}\PY{n}{i}\PY{o}{+}\PY{l+m+mi}{1}\PY{p}{,}\PY{n}{y}\PY{p}{)}    
        
        \PY{c+c1}{\PYZsh{}tester avec print(i,y), et constater que le premier terme est alors y\PYZus{}0}
\end{Verbatim}

    \begin{Verbatim}[commandchars=\\\{\}]
1 -3
2 -1
3 3
4 9
5 17
6 27
7 39
8 53
9 69
10 87

    \end{Verbatim}

    \subsection{2. Utiliser range(début, fin,
pas)}\label{utiliser-rangeduxe9but-fin-pas}

\begin{quote}
Ajoutez les cellules (menu du haut : bouton "+"), et testez les
programmes de la fiche de TD.\\
Améliorez l'affichage avec
\texttt{print(\textquotesingle{}i=\textquotesingle{},i)}
\end{quote}

    \begin{Verbatim}[commandchars=\\\{\}]
{\color{incolor}In [{\color{incolor}3}]:} \PY{k}{for} \PY{n}{i} \PY{o+ow}{in} \PY{n+nb}{range}\PY{p}{(}\PY{l+m+mi}{10}\PY{p}{)} \PY{p}{:}
            \PY{n+nb}{print}\PY{p}{(}\PY{l+s+s1}{\PYZsq{}}\PY{l+s+s1}{i=}\PY{l+s+s1}{\PYZsq{}}\PY{p}{,}\PY{n}{i}\PY{p}{)}   \PY{c+c1}{\PYZsh{} j\PYZsq{}ajoute \PYZsq{}i =\PYZsq{}, qui est une chaîne de caractères}
\end{Verbatim}

    \begin{Verbatim}[commandchars=\\\{\}]
i= 0
i= 1
i= 2
i= 3
i= 4
i= 5
i= 6
i= 7
i= 8
i= 9

    \end{Verbatim}

    \begin{Verbatim}[commandchars=\\\{\}]
{\color{incolor}In [{\color{incolor}4}]:} \PY{k}{for} \PY{n}{i} \PY{o+ow}{in} \PY{n+nb}{range}\PY{p}{(}\PY{l+m+mi}{5}\PY{p}{,}\PY{l+m+mi}{10}\PY{p}{)} \PY{p}{:}
            \PY{n+nb}{print}\PY{p}{(}\PY{l+s+s1}{\PYZsq{}}\PY{l+s+s1}{i=}\PY{l+s+s1}{\PYZsq{}}\PY{p}{,}\PY{n}{i}\PY{p}{)}
\end{Verbatim}

    \begin{Verbatim}[commandchars=\\\{\}]
i= 5
i= 6
i= 7
i= 8
i= 9

    \end{Verbatim}

    \begin{Verbatim}[commandchars=\\\{\}]
{\color{incolor}In [{\color{incolor}5}]:} \PY{k}{for} \PY{n}{i} \PY{o+ow}{in} \PY{n+nb}{range}\PY{p}{(}\PY{l+m+mi}{5}\PY{p}{,}\PY{l+m+mi}{10}\PY{p}{,}\PY{l+m+mi}{2}\PY{p}{)} \PY{p}{:}
            \PY{n+nb}{print}\PY{p}{(}\PY{l+s+s1}{\PYZsq{}}\PY{l+s+s1}{i=}\PY{l+s+s1}{\PYZsq{}}\PY{p}{,}\PY{n}{i}\PY{p}{)}
\end{Verbatim}

    \begin{Verbatim}[commandchars=\\\{\}]
i= 5
i= 7
i= 9

    \end{Verbatim}

    \subsection{3. Comportement à l'infini}\label{comportement-uxe0-linfini}

    \subsection{\texorpdfstring{(\textbf{3.1.}) Suite
\((w_n)\)}{(3.1.) Suite (w\_n)}}\label{suite-w_n}

\subsubsection{\texorpdfstring{(\textbf{3.1.1.})}{(3.1.1.)}}\label{section}

\begin{quote}
Utilisez la cellule ci-dessous pour calculer les termes de 10000 à
100000, avec un pas de 20000, de la suite \((w_n)\). Testez les deux
programmes, changez les valeurs du \texttt{range} à votre guise pour
pouvoir répondre aux questions qui suivent.
\end{quote}

    \begin{Verbatim}[commandchars=\\\{\}]
{\color{incolor}In [{\color{incolor}6}]:} \PY{k}{for} \PY{n}{i} \PY{o+ow}{in} \PY{n+nb}{range} \PY{p}{(}\PY{l+m+mi}{10000}\PY{p}{,}\PY{l+m+mi}{100000}\PY{p}{,}\PY{l+m+mi}{20000}\PY{p}{)}\PY{p}{:}
            \PY{n}{w}\PY{o}{=}\PY{p}{(}\PY{l+m+mi}{2}\PY{o}{*}\PY{n}{i}\PY{o}{\PYZhy{}}\PY{l+m+mi}{3}\PY{p}{)}\PY{o}{/}\PY{p}{(}\PY{n}{i}\PY{o}{+}\PY{l+m+mi}{2}\PY{p}{)}
            \PY{n+nb}{print}\PY{p}{(}\PY{n}{i}\PY{p}{,}\PY{n}{w}\PY{p}{)}   \PY{c+c1}{\PYZsh{}j\PYZsq{}obtiens les termes w\PYZus{}n , pour n qui vaut 10000, 30000,..., 90000}
\end{Verbatim}

    \begin{Verbatim}[commandchars=\\\{\}]
10000 1.9993001399720056
30000 1.9997666822211853
50000 1.999860005599776
70000 1.9999000028570613
90000 1.999922223950579

    \end{Verbatim}

    \paragraph{\texorpdfstring{Question : savez-vous si la suite \((w_n)\)
est monotone ? Si oui, précisez. Peut-on le prouver, ou l'avons-nous
déjà prouvé en classe
?}{Question : savez-vous si la suite (w\_n) est monotone ? Si oui, précisez. Peut-on le prouver, ou l'avons-nous déjà prouvé en classe ?}}\label{question-savez-vous-si-la-suite-w_n-est-monotone-si-oui-pruxe9cisez.-peut-on-le-prouver-ou-lavons-nous-duxe9juxe0-prouvuxe9-en-classe}

\begin{quote}
Faire \texttt{entrée} pour entrer dans cette cellule, puis répondre,
puis \texttt{Ctrl\ entrée} pour en sortir. \textgreater{} Réponse :
\end{quote}

    Nous avons prouvé en classe que la suite \((w_n)\) est croissante.

    \paragraph{\texorpdfstring{Question : y a-t-il une valeur que la suite
\((w_n)\) semble ne pas dépasser
?}{Question : y a-t-il une valeur que la suite (w\_n) semble ne pas dépasser ?}}\label{question-y-a-t-il-une-valeur-que-la-suite-w_n-semble-ne-pas-duxe9passer}

\begin{quote}
Faire \texttt{entrée} pour entrer dans cette cellule, puis répondre,
puis \texttt{Ctrl\ entrée} pour en sortir. \textgreater{} Réponse :
\end{quote}

    Il semble qu'aucun des termes de la suite \((w_n)\) ne dépasse ni
n'atteint 2.

    \subsubsection{\texorpdfstring{(\textbf{3.1.2.}) Recherche de
seuils}{(3.1.2.) Recherche de seuils}}\label{recherche-de-seuils}

\begin{quote}
Il y a une petite erreur dans le programme que je vous ai proposé (page
3). Corrigez-le et composez-le dans la cellule ci-dessous : le but est
d'obtenir les valeurs proposées.
\end{quote}

    \begin{Verbatim}[commandchars=\\\{\}]
{\color{incolor}In [{\color{incolor}7}]:} \PY{k}{for} \PY{n}{i} \PY{o+ow}{in} \PY{n+nb}{range}\PY{p}{(}\PY{l+m+mi}{10}\PY{p}{)}\PY{p}{:}
            \PY{n}{e}\PY{o}{=}\PY{l+m+mi}{2}\PY{o}{\PYZhy{}}\PY{p}{(}\PY{l+m+mi}{2}\PY{o}{*}\PY{n}{i}\PY{o}{\PYZhy{}}\PY{l+m+mi}{3}\PY{p}{)}\PY{o}{/}\PY{p}{(}\PY{n}{i}\PY{o}{+}\PY{l+m+mi}{2}\PY{p}{)}
            \PY{n+nb}{print}\PY{p}{(}\PY{n}{i}\PY{p}{,}\PY{n}{e}\PY{p}{)}
\end{Verbatim}

    \begin{Verbatim}[commandchars=\\\{\}]
0 3.5
1 2.3333333333333335
2 1.75
3 1.4
4 1.1666666666666665
5 1.0
6 0.875
7 0.7777777777777777
8 0.7
9 0.6363636363636365

    \end{Verbatim}

    \begin{quote}
\textbf{À chaque itération \texttt{i}, la variable \texttt{e}est l'écart
entre \(y_i\) et 2.} Vous comprenez ?\\
Comparez dans la cellule suivante les deux programmes de la page 4, et
répondez à la question sur le document. (Ici il est important que
\texttt{i=2}pour démarrer, et nous l'expliquerons plus tard).
\end{quote}

    \begin{Verbatim}[commandchars=\\\{\}]
{\color{incolor}In [{\color{incolor}8}]:} \PY{n}{i}\PY{o}{=}\PY{l+m+mi}{2}
        \PY{n}{e}\PY{o}{=}\PY{l+m+mi}{2}\PY{o}{\PYZhy{}}\PY{p}{(}\PY{l+m+mi}{2}\PY{o}{*}\PY{n}{i}\PY{o}{\PYZhy{}}\PY{l+m+mi}{3}\PY{p}{)}\PY{o}{/}\PY{p}{(}\PY{n}{i}\PY{o}{+}\PY{l+m+mi}{2}\PY{p}{)}
        \PY{k}{while} \PY{n}{e}\PY{o}{\PYZgt{}}\PY{o}{=}\PY{l+m+mf}{0.5}\PY{p}{:}
            \PY{n}{i}\PY{o}{=}\PY{n}{i}\PY{o}{+}\PY{l+m+mi}{1}
            \PY{n}{e}\PY{o}{=}\PY{l+m+mi}{2}\PY{o}{\PYZhy{}}\PY{p}{(}\PY{l+m+mi}{2}\PY{o}{*}\PY{n}{i}\PY{o}{\PYZhy{}}\PY{l+m+mi}{3}\PY{p}{)}\PY{o}{/}\PY{p}{(}\PY{n}{i}\PY{o}{+}\PY{l+m+mi}{2}\PY{p}{)}
            \PY{n+nb}{print}\PY{p}{(}\PY{n}{i}\PY{p}{,}\PY{n}{e}\PY{p}{)}
\end{Verbatim}

    \begin{Verbatim}[commandchars=\\\{\}]
3 1.4
4 1.1666666666666665
5 1.0
6 0.875
7 0.7777777777777777
8 0.7
9 0.6363636363636365
10 0.5833333333333333
11 0.5384615384615385
12 0.5
13 0.46666666666666656

    \end{Verbatim}

    \subsubsection{À propos de l'algorithme demadé page
4.}\label{uxe0-propos-de-lalgorithme-demaduxe9-page-4.}

\begin{quote}
Composez dans la cellule ci-dessous un programme qui permet de
déterminer à partir de quel entier \(n_0\) l'écart \(e\) sera tel que :
\(0<e<0.001\).\\
\textgreater{} Réponse : \(n_0=\) ...........
\end{quote}

    \begin{Verbatim}[commandchars=\\\{\}]
{\color{incolor}In [{\color{incolor}9}]:} \PY{n}{i}\PY{o}{=}\PY{l+m+mi}{2}
        \PY{n}{e}\PY{o}{=}\PY{l+m+mi}{2}\PY{o}{\PYZhy{}}\PY{p}{(}\PY{l+m+mi}{2}\PY{o}{*}\PY{n}{i}\PY{o}{\PYZhy{}}\PY{l+m+mi}{3}\PY{p}{)}\PY{o}{/}\PY{p}{(}\PY{n}{i}\PY{o}{+}\PY{l+m+mi}{2}\PY{p}{)}
        \PY{k}{while} \PY{n}{e}\PY{o}{\PYZgt{}}\PY{o}{=}\PY{l+m+mf}{0.001}\PY{p}{:}
            \PY{n}{i}\PY{o}{=}\PY{n}{i}\PY{o}{+}\PY{l+m+mi}{1}
            \PY{n}{e}\PY{o}{=}\PY{l+m+mi}{2}\PY{o}{\PYZhy{}}\PY{p}{(}\PY{l+m+mi}{2}\PY{o}{*}\PY{n}{i}\PY{o}{\PYZhy{}}\PY{l+m+mi}{3}\PY{p}{)}\PY{o}{/}\PY{p}{(}\PY{n}{i}\PY{o}{+}\PY{l+m+mi}{2}\PY{p}{)}
        \PY{n+nb}{print}\PY{p}{(}\PY{n}{i}\PY{p}{,}\PY{n}{e}\PY{p}{)}
\end{Verbatim}

    \begin{Verbatim}[commandchars=\\\{\}]
6998 0.0009999999999998899

    \end{Verbatim}

    \subsection{\texorpdfstring{(\textbf{3.2.}) Suite
\((y_n)\)}{(3.2.) Suite (y\_n)}}\label{suite-y_n}

\begin{quote}
À votre tour !
\end{quote}

    \begin{Verbatim}[commandchars=\\\{\}]
{\color{incolor}In [{\color{incolor}10}]:} \PY{n}{y}\PY{o}{=}\PY{o}{\PYZhy{}}\PY{l+m+mi}{3}
         \PY{k}{for} \PY{n}{i} \PY{o+ow}{in} \PY{n+nb}{range}\PY{p}{(}\PY{l+m+mi}{10}\PY{p}{)}\PY{p}{:}
             \PY{n}{y}\PY{o}{=}\PY{n}{y}\PY{o}{+}\PY{l+m+mi}{2}\PY{o}{*}\PY{n}{i}
             \PY{n+nb}{print}\PY{p}{(}\PY{n}{i}\PY{o}{+}\PY{l+m+mi}{1}\PY{p}{,}\PY{n}{y}\PY{p}{)}
\end{Verbatim}

    \begin{Verbatim}[commandchars=\\\{\}]
1 -3
2 -1
3 3
4 9
5 17
6 27
7 39
8 53
9 69
10 87

    \end{Verbatim}

    Nous avons montré en classe que la suite \(y_n\) est croissante. Il
semble qu'elle n'ait pas de valeur "d'arrêt", que ses termes croissent
indéfiniment.

    \begin{Verbatim}[commandchars=\\\{\}]
{\color{incolor}In [{\color{incolor}11}]:} \PY{n}{y}\PY{o}{=}\PY{o}{\PYZhy{}}\PY{l+m+mi}{3}
         \PY{n}{i}\PY{o}{=}\PY{l+m+mi}{1}
         \PY{k}{while} \PY{n}{y}\PY{o}{\PYZlt{}}\PY{o}{=}\PY{l+m+mi}{20} \PY{p}{:}
             \PY{n}{y}\PY{o}{=}\PY{n}{y}\PY{o}{+}\PY{l+m+mi}{2}\PY{o}{*}\PY{n}{i}
             \PY{n}{i}\PY{o}{=}\PY{n}{i}\PY{o}{+}\PY{l+m+mi}{1}
             \PY{n+nb}{print}\PY{p}{(}\PY{n}{i}\PY{p}{,}\PY{n}{y}\PY{p}{)}
\end{Verbatim}

    \begin{Verbatim}[commandchars=\\\{\}]
2 -1
3 3
4 9
5 17
6 27

    \end{Verbatim}

    \begin{Verbatim}[commandchars=\\\{\}]
{\color{incolor}In [{\color{incolor}12}]:} \PY{n}{y}\PY{o}{=}\PY{o}{\PYZhy{}}\PY{l+m+mi}{3}
         \PY{n}{i}\PY{o}{=}\PY{l+m+mi}{1}
         \PY{k}{while} \PY{n}{y}\PY{o}{\PYZlt{}}\PY{o}{=}\PY{l+m+mi}{20} \PY{p}{:}
             \PY{n}{y}\PY{o}{=}\PY{n}{y}\PY{o}{+}\PY{l+m+mi}{2}\PY{o}{*}\PY{n}{i}
             \PY{n}{i}\PY{o}{=}\PY{n}{i}\PY{o}{+}\PY{l+m+mi}{1}
         \PY{n+nb}{print}\PY{p}{(}\PY{n}{i}\PY{p}{,}\PY{n}{y}\PY{p}{)}
\end{Verbatim}

    \begin{Verbatim}[commandchars=\\\{\}]
6 27

    \end{Verbatim}

    \begin{Verbatim}[commandchars=\\\{\}]
{\color{incolor}In [{\color{incolor}13}]:} \PY{n}{y}\PY{o}{=}\PY{o}{\PYZhy{}}\PY{l+m+mi}{3}
         \PY{n}{i}\PY{o}{=}\PY{l+m+mi}{1}
         \PY{k}{while} \PY{n}{y}\PY{o}{\PYZlt{}}\PY{o}{=}\PY{l+m+mi}{10}\PY{o}{*}\PY{o}{*}\PY{l+m+mi}{9} \PY{p}{:}
             \PY{n}{y}\PY{o}{=}\PY{n}{y}\PY{o}{+}\PY{l+m+mi}{2}\PY{o}{*}\PY{n}{i}
             \PY{n}{i}\PY{o}{=}\PY{n}{i}\PY{o}{+}\PY{l+m+mi}{1}
         \PY{n+nb}{print}\PY{p}{(}\PY{n}{i}\PY{p}{,}\PY{n}{y}\PY{p}{)}
\end{Verbatim}

    \begin{Verbatim}[commandchars=\\\{\}]
31624 1000045749

    \end{Verbatim}

    \begin{Verbatim}[commandchars=\\\{\}]
{\color{incolor}In [{\color{incolor}14}]:} \PY{n}{y}\PY{o}{=}\PY{o}{\PYZhy{}}\PY{l+m+mi}{3}
         \PY{n}{i}\PY{o}{=}\PY{l+m+mi}{1}
         \PY{k}{while} \PY{n}{y}\PY{o}{\PYZlt{}}\PY{o}{=}\PY{l+m+mi}{10}\PY{o}{*}\PY{o}{*}\PY{l+m+mi}{12} \PY{p}{:}
             \PY{n}{y}\PY{o}{=}\PY{n}{y}\PY{o}{+}\PY{l+m+mi}{2}\PY{o}{*}\PY{n}{i}
             \PY{n}{i}\PY{o}{=}\PY{n}{i}\PY{o}{+}\PY{l+m+mi}{1}
         \PY{n+nb}{print}\PY{p}{(}\PY{n}{i}\PY{p}{,}\PY{n}{y}\PY{p}{)}
\end{Verbatim}

    \begin{Verbatim}[commandchars=\\\{\}]
1000001 1000000999997

    \end{Verbatim}

    Réponses :\\
\textgreater{} à partir du rang \(6\), tous les termes de la suite
\((y_n)\) sont strictement supérieurs à \(1000\).\\
\textgreater{} à partir du rang \(31624\), tous les termes de la suite
\((y_n)\) sont strictement supérieurs à \(10^{9}\).\\
\textgreater{} à partir du rang \(1000001\), tous les termes de la suite
\((y_n)\) sont strictement supérieurs à \(10^{12}\).


    % Add a bibliography block to the postdoc
    
    
    
    \end{document}
